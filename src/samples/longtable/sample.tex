\documentclass[a4j,11pt,papersize]{jsarticle}
\usepackage{longtable}
\newcommand\hoge[1][0]{%
  醤油 #1-0 & 32892378923894832894 & 1000 \\%
  醤油 #1-1 & 32892378923894832894 & 1000 \\%
  醤油 #1-2 & 32892378923894832894 & 1000 \\%
  醤油 #1-3 & 32892378923894832894 & 1000 \\%
  醤油 #1-4 & 32892378923894832894 & 1000 \\%
  醤油 #1-5 & 32892378923894832894 & 1000 \\%
  醤油 #1-6 & 32892378923894832894 & 1000 \\%
  醤油 #1-7 & 32892378923894832894 & 1000 \\%
  醤油 #1-8 & 32892378923894832894 & 1000 \\%
  醤油 #1-9 & 32892378923894832894 & 1000 \\%
}
\begin{document}
% 表の幅を取得するために \jobname.aux に longtable パッケージは
% 情報を書き出し、2 回目以上のタイプセットで幅をそろえる。
%
\newcommand\mytablehead{\hline 商品 & 番号 & 個数 \\}
\begin{longtable}{|l|l|l|}
% 表の最初のページの上部にだけ表示する行の終わり
\caption{長いながーい表\label{tab:longtable}}
\endfirsthead
%
% 行がページを跨ぐとき、各ページの上部に表示する行の終わり
\hline
\multicolumn{3}{|c|}{前ページの表の続きです。}\\
\mytablehead
\hline
\endhead
%
% 行がページを跨ぐとき、各ページの下部に表示する行の終わり
\hline
\multicolumn{3}{|c|}{この表の続きが次ページにあります。}\\
\hline
\endfoot
%
% 表の最後のページの下部だけに表示する行の終わり
\multicolumn{3}{|c|}{これでこの表は終わりです。}\\
\hline
\endlastfoot
%
% 実際の表の始まり
\mytablehead
\hline
\hoge[1]
\hoge[2]
\hoge[3]
\hoge[4]
\hoge[5]
\hoge[6]
\hoge[7]
\hoge[8]
\hoge[9]
\hoge[10]
\hoge[11]
\hoge[12]
\hoge[13]
\hline
\end{longtable}
\end{document}
