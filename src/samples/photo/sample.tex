\documentclass[a4j,11pt,papersize,draft]{jsarticle}
\usepackage[shortlop,under]{photo}
\def\photoname{写真}
\def\listphotoname{写真一覧}
\oecaptionsep  0pt
\parskip   = 1zw
\newcommand\myphoto{\fbox{ここに写真がはいるはず\rule{30zw}{0pt}}}
\begin{document}
\parindent = 0zw
%
\listofphotos\clearpage
%
\begin{photo}[htbp]
 \begin{center}
  \myphoto% 写真の要素
  \caption{写真の見出し\label{ph:hoge}}
 \end{center}
\end{photo}
%
\putphoto{ph:foo}{}{\myphoto}[これはちょっとした例]{これはちょっとした例なんだ けど、
見出しがながーくなると自動的にあっちにふらふら}
\par
\phref{ph:hoge}とか \Phref{ph:foo} などなど。
\par
\putphoto[icu]{ph:piyo}{}{\myphoto}{位置を指定する}
\par
\begin{Photo}{ph:bar}{}{これも例}
 \myphoto
\end{Photo}
\renewcommand\myphoto{\fbox{ここに写真がはいる\rule{0pt}{9zw}}}
\par\noindent\putphoto[l]{ph:lhoge}{}{\myphoto}{位置を指定する}
\par\noindent\putphoto[r]{ph:rhoge}{}{\myphoto}{位置を指定する}
\end{document}
