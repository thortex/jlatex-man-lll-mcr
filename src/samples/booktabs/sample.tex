\documentclass[a4j,11pt,papersize]{jsarticle}
\usepackage{booktabs}
\begin{document}

\begin{table}[htbp]
 \begin{center}
  \caption{日本人ごのみ}
  \begin{tabular}{|l||l|l|}
   \hline
   名称   & 型番  & 個数 \\
   \hline \hline
   たわし & TWS01 & 1000 \\
   \hline
   石鹸   & SP01  & 5000 \\
   \hline
  \end{tabular}
 \end{center}
\end{table}

\begin{table}[htbp]
\begin{center}
 \caption{スマートでイージー}
 \begin{tabular}{lll}
 \hline
 名称   & 型番  & 個数 \\
 \hline
 たわし & TWS01 & 1000 \\
 石鹸   & SP01  & 5000 \\
 \hline
\end{tabular}
\end{center} 
\end{table}

\begin{table}[htbp]
 \begin{center}
  \caption{\textsf{booktabs}を使った例}
  \begin{tabular}{lll}
   \toprule
   品名 & 番号 & 個数 \\
   \midrule
   たわし & 02A & 3 \\
   雑巾   & 55B & 2 \\
   傘     & X2B & 5 \\
   \bottomrule
  \end{tabular}
 \end{center} 
\end{table}

\begin{table}[htbp]
 \begin{center}
  \caption{中罫線がある場合}
  \begin{tabular}{lll}
   \toprule
   \multicolumn{2}{c}{項目} & \\
   \cmidrule{1-2}
   品名 & 型番 & 個数\\
   \midrule
   たわし & 02A & 3 \\
   雑巾   & 55B & 2 \\
   傘     & X2B & 5 \\
   \bottomrule
  \end{tabular}
 \end{center}
\end{table}


\end{document}