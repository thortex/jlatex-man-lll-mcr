\chapter{まえがき}

\section*{私の舟では \TeX という大海原を航海できない事は明白である}

CTAN などを航海すれば既存のたくさんの素晴しいマクロに出会うことが出来ま
す.しかし,現在の CTAN は初心者にとってはどこから手を付けて良いのか
検討も付かないような大海原です.ここで少しは助け船を出すことまでは
出来ませんが,ほんの僅かな光を灯す灯台くらいにはなれると思います.

この冊子では既存のマクロ/クラスの構造を分析するわけではなく,単なるユー
ザーとしてうまく活用することを考えます.コンパニオンシリーズ的な面が強い
と思われます.自分でマクロを作成/編集という内容は中級編以上のシリーズを
読んでください.


\section*{集団の総意に基づいて航海は出来ない}

大海原を航海する船に船長はたった一人で良いことは明白です.二人以上いると
行き先が決まりません.

このような冊子の作成においてもそれは同様で,一人で作る場合と集団で執筆す
るのとでは方向性や手法が異なります.そこで,私はこの冊子を基本的に私の独
断で編集することにしています.読者の方からの意見や要望,質問等は大歓迎で
すが,市場の需要や経済性等はほとんど考慮していません.要するに趣味で提供
している代物なので,明後日の方向に進んでいます.しかし,現在書店に並んで
いる \LaTeX 絡みのハウツー本 (適切な表現とは言いがたいが) よりはクオリティ
の高いものを提供しようと考えています (『だれでもできる○○』とか,そうい
う本を良く見掛けますが,\LaTeX はそんなに浅いプログラムではないと思って
います).\laTEX は噛めば噛む程その味が分かってきます.是非とも長旅の準備
をして,この大航海につき合っていただきたいものです.

\section*{磁石のない航海ほど危険なものはない}

私が趣味で執筆している冊子なので,明後日の方向に進んでいるのは勘弁してい
ただいて,一応この冊子もコンパスをもって航海することにしています.その
方向は『初心者が書籍作成やレポート/論文作成の段階でおそらく必要になるだろ
うマクロパッケージの抜粋』にあります.


\section*{凡例}

本冊子では\Z{書体}を変更することによって同じ語句でも
違った意味を持つものが多数あります.\qu{\prog{dvipdfm}}
という語があったとしても\qu{\textsf{dvipdfm}}や\qu{\texttt{dvipdfm}},
\qu{\textsl{dvipdfm}},\qu{\textit{dvipdfm}}はすべて別の意味を
持っています.\zindind{書体}{の種類}
%これらの書体の種類については\secref{sec:font}を参
%照してください.
\begin{center}
\zindind{キーボード}{からの入力}%
 \begin{tabular}{lll}
\hline
 書体          & 意味      & 例 \\
\hline
 ローマン体    & 通常の文章& \textrm{dvipdfm}\\
 サンセリフ体  & パッケージやクラス& \textsf{dvipdfm}\\
 タイプライタ体& キーボードからの入力など& \texttt{dvipdfm}\\
 イタリック体  & 変数や強調& \textit{dvipdfm}\\
 スラント体    & オプション& \textsl{dvipdfm}\\
\hline
 \end{tabular}
\end{center}

%本文中で左側にタイプライタ体,右側にそれに準じた出力例が
%あるものは,入出力の対を表します.
%\par\addvspace{3.0ex plus 0.8ex minus 0.5ex}\vskip-\parskip%
%\hspace*{\IOm}\hspace*{-1ex}%
%\makebox[0pt][l]{%
%  {\begin{minipage}[c]{.47\fullwidth}\small%
%\begin{ttfamily}%
%The length of a pen should be comrotable\\
%to write with: too long and it makes\\
%him tired; too short and it\cmd{ldots}.%
%\end{ttfamily}%
%  \end{minipage}%
%  }\hspace{0.05\fullwidth}%
%  {\begin{minipage}{.47\fullwidth}%
%      \begin{trivlist}\item\small%
%The length of a pen should be comrotable
%to write with: too long and it makes him
%tired; too short and it\ldots.%
%   \end{trivlist}%
%   \end{minipage}}}%
%\par\addvspace{3.0ex plus 0.8ex minus 0.5ex}\vskip-\parskip%
%
%\Z{テキストエディッタ}などを使い,原稿ファイルで
%左側のように入力すると,右側の出力例と同じような
%結果を確認できます.

\bigskip

文中において\type{which perl}という表記は\Z{コマンドプロンプト}や
\Z{シェル}などの\Z{コンソール}からの入力を示します.
複数行の入力の場合は次のようにしています.
\begin{InTerm}
\type{platex file.tex} 
\type{jbibtex file.tex} 
\type{dvipdfmx -S -o out.pdf input.dvi}
\end{InTerm}
\index{"$@\verb+$+!コンソールの\zdash}%"
先頭のドル`\str$'はコンソールに表示されている記号で,
ユーザは入力しません.

\bigskip

\Z{キーボード}上の特定の\Z{キートップ}を押すことを示すには\key{Alt}の
ようにします.\key{Ctrl,Alt,Delete}は\key{Ctrl},\key{Alt},
\key{Delete}キーを同時に押すことになります.
\key{Ctrl,x}\key{Ctrl,s}は\key{Ctrl,x}を押した後に\key{Ctrl,s}を
押すことを表します.

\bigskip

何らかの文字列や数値に置き換わるものは\va{変数}のように
表記しています.

\endinput

\section*{凡例}

本冊子では書体を変更することによって同じ語句でも
違った意味を持つものが多数あります.\qu{\prog{dvips}}
という語があったとしても\qu{\textsf{dvips}}や
\qu{\texttt{dvips}},\qu{\textsl{dvips}},\qu{\textit{dvips}}
はすべて別の意味を持っています.
%これらの書体の種類については\secref{sec:font}を参照してください.
\begin{center}
 \begin{tabular}{lll}
 \hline
 書体          & 意味      & 例\\
 \hline
 ローマン体    & 通常の文章& \textrm{dvips}\\
 サンセリフ体  & パッケージやクラス& \textsf{dvips}\\
 タイプライタ体& キーボードからの入力など& \texttt{dvips}\\
 イタリック体  & 変数や強調& \textit{dvips}\\
 スラント体    & オプション& \textsl{dvips}\\
 \hline
 \end{tabular}
\end{center}





